\documentclass{amsart}
\usepackage{amsmath,amssymb,latexsym}
\usepackage[margin=1in]{geometry}
\usepackage{tikz}

\newtheorem{theorem}{Theorem}
\newtheorem{definition}[theorem]{Definition}
\newtheorem{corollary}[theorem]{Corollary}
\newtheorem{example}[theorem]{Example}
\newtheorem{lemma}[theorem]{Lemma}
\newtheorem{proposition}[theorem]{Proposition}

\newcommand{\bepsilon}{\boldsymbol{\epsilon}}
\newcommand{\fsl}{\mathfrak{sl}}
\newcommand{\ZZ}{\mathbb{Z}}

\makeatletter
\renewcommand{\@evenhead}{\tiny\hfill $\fsl_2$-Tilings Do Not Exist in Higher Dimensions (mostly)\hfill \thepage}

\title{$\fsl_2$-Tilings Do Not Exist in Higher Dimensions (mostly)}

\begin{document}
  \begin{abstract}
    We show that there do not exist $\fsl_2$ tilings in dimension 3 or higher and that there exists a unique anti-$\fsl_2$ tiling in each dimension beyond 2.
  \end{abstract}
  \maketitle

  \section{$\fsl_2$-Tilings of the Plane}
    \begin{definition}
      A bi-infinite array $(a_{ij})_{i,j\in\ZZ}$ with $a_{ij}\in\ZZ_{>0}$ is called an \emph{$\fsl_2$-tiling of $\ZZ^2$} if the entries satisfy the relation
      \begin{equation}\label{eq:sl2 recursion}
        a_{ij}a_{i+1,j-1}-a_{i+1,j}a_{i,j-1}=1.
      \end{equation}
      A bi-infinite array $(b_{ij})_{i,j\in\ZZ}$ with $b_{ij}\in\ZZ_{>0}$  is called an \emph{anti-$\fsl_2$-tiling of $\ZZ^2$} if the entries satisfy the relation
      \begin{equation}\label{eq:anti-sl2 recursion}
        b_{ij}b_{i+1,j-1}-b_{i,j-1}b_{i+1,j}=-1.
      \end{equation}
    \end{definition}
    The notion of an anti-$\fsl_2$-tiling is not actually giving anything new as shown by the following lemma, however this notion will be useful for our considerations in higher dimensions.
    \begin{lemma}
      If $(a_{ij})_{i,j\in\ZZ}$ is an $\fsl_2$-tiling, then taking $b_{ij}=a_{i,-j}$ gives an anti-$\fsl_2$-tiling.
    \end{lemma}
    One should think of the difference between $\fsl_2$-tilings and anti-$\fsl_2$-tilings as viewing the lattice $\ZZ^2$ ``from above'' or ``from below.''
    \begin{theorem}
      There exist infinitely many $\fsl_2$ tilings of $\ZZ^2$.
    \end{theorem}

    \begin{example}\label{ex:Fibonacci}
      Consider the $\fsl_2$-tiling $(a_{ij})_{i,j\in\ZZ}$ of $\ZZ^2$ with $a_{ij}=1$ if $i-j\in\{0,1\}$.  Then using \eqref{eq:sl2 recursion} and the well-known recursion $F_{2r-1}F_{2r+3}=F_{2r+1}^2+1$ $(r\ge1)$ for the odd Fibonacci numbers, we see that
      \[a_{ij}=\begin{cases}F_{2r-1} & \text{if $i-j=r\ge1$;}\\F_{-2r+1} & \text{if $i-j=r\le0$;}\end{cases}\]
      where we number the Fibonacci numbers as:
      \[\begin{tabular}{|c|c|c|c|c|c|c|c} $F_1$ & $F_2$ & $F_3$ & $F_4$ & $F_5$ & $F_6$ & $F_7$ & $\cdots$\\\hline 1 & 1 & 2 & 3 & 5 & 8 & 13 & $\cdots$\end{tabular}\]
    \end{example}

  \section{$\fsl_2$-Tilings in Higher Dimensions}
    \begin{definition}
      Fix $\bepsilon=(\epsilon_{k\ell})_{1\le k<\ell\le n}\in\{\pm1\}^{{n\choose 2}}$.  An array $(a_{i_1,\ldots,i_n})_{i_k\in\ZZ}$ with $a_{i_1,\ldots,i_n}\in\ZZ_{>0}$ is called an \emph{$\bepsilon-\fsl_2$-tiling of $\ZZ^n$} if for each $1\le k<\ell\le n$ we have
      \begin{equation}\label{eq:higher sl2 recursion}
        a_{i_1,\ldots,i_k,\ldots,i_\ell+1,\ldots,i_n}a_{i_1,\ldots,i_k+1,\ldots,i_\ell,\ldots,i_n}-a_{i_1,\ldots,i_k,\ldots,i_\ell,\ldots,i_n}a_{i_1,\ldots,i_k+1,\ldots,i_\ell+1,\ldots,i_n}=\epsilon_{k\ell}.
      \end{equation}
    \end{definition}
    The situation is now different than the $n=2$ case, all the $\bepsilon$-$\fsl_2$-tilings are not necessarily equivalent, however there do remain relations among them.
    \begin{lemma}
      Let $\bepsilon=(\epsilon_{k\ell})_{1\le k<\ell\le n}\in\{\pm1\}^{{n\choose 2}}$ and write $\bepsilon^{(r)}=(\epsilon'_{k\ell})_{1\le k<\ell\le n}\in\{\pm1\}^{{n\choose 2}}$ where $\epsilon'_{k\ell}=-\epsilon_{k\ell}$ if $k=r$ or $\ell=r$ and $\epsilon'_{k\ell}=\epsilon_{k\ell}$ otherwise.  If $(a_{i_1,\ldots,i_n})_{i_k\in\ZZ}$ is an $\bepsilon-\fsl_2$-tiling, then taking $b_{i_1,\ldots,i_n}=a_{i_1,\ldots,-i_r,\ldots,i_n}$ gives an $\bepsilon^{(r)}-\fsl_2$-tiling.
    \end{lemma}
    \begin{lemma}\label{le:constant slices}
      Let $\epsilon=\pm1$ and assume $(a_{i_1,i_2,i_3})_{i_k\in\ZZ}$ is an $(\epsilon,\epsilon,\epsilon)-\fsl_2$-tiling of $\ZZ^3$.  Then for any $r\in\ZZ$ the set $\{a_{i_1,i_2,i_3}:i_1+i_2+i_3=r\}$ consists of a single element.
    \end{lemma}
    \begin{proof}
      To see this we compute $a_{i_1+1,i_2+1,i_3+1}$ in terms of $a_{i_1,i_2,i_3}, a_{i_1+1,i_2,i_3}, a_{i_1,i_2+1,i_3}, a_{i_1,i_2,i_3+1}$ in three different ways.  For simplicity of notation we set:
      \[a_{i_1,i_2,i_3}=a,\quad\quad a_{i_1+1,i_2,i_3}=x,\quad\quad a_{i_1,i_2+1,i_3}=y,\quad\quad a_{i_1,i_2,i_3+1}=z.\]
      Using \eqref{eq:higher sl2 recursion} three times we get
      \[a_{i_1+1,i_2+1,i_3}=\frac{xy-\epsilon}{a},\quad\quad a_{i_1,i_2+1,i_3+1}=\frac{yz-\epsilon}{a},\quad\quad a_{i_1+1,i_2,i_3+1}=\frac{xz-\epsilon}{a}.\]
      Then applying \eqref{eq:higher sl2 recursion} three more times gives
      \begin{align*}
        a_{i_1+1,i_2+1,i_3+1} 
        &= \frac{a_{i_1+1,i_2+1,i_3}a_{i_1+1,i_2,i_3+1}-\epsilon}{a_{i_1+1,i_2,i_3}}=\frac{xyz}{a^2}-\epsilon\frac{y+z}{a^2}-\epsilon\frac{a^2-\epsilon}{a^2x};\\
        &= \frac{a_{i_1+1,i_2+1,i_3}a_{i_1,i_2+1,i_3+1}-\epsilon}{a_{i_1,i_2+1,i_3}}=\frac{xyz}{a^2}-\epsilon\frac{x+z}{a^2}-\epsilon\frac{a^2-\epsilon}{a^2y};\\
        &= \frac{a_{i_1+1,i_2,i_3+1}a_{i_1,i_2+1,i_3+1}-\epsilon}{a_{i_1,i_2,i_3+1}}=\frac{xyz}{a^2}-\epsilon\frac{x+y}{a^2}-\epsilon\frac{a^2-\epsilon}{a^2z}.
      \end{align*}
      It follows that $\frac{x-y}{a^2}=\frac{a^2-\epsilon}{a^2x}-\frac{a^2-\epsilon}{a^2y}$ or $(xy+a^2-\epsilon)(x-y)=0$.  But $xy+a^2-\epsilon\ge1$ since $a,x,y\ge1$, hence $x=y$.  Similarly $y=z$ and the result follows.
    \end{proof}

    \begin{theorem}
      For $\bepsilon=(-1,-1,-1)$, there exists a unique (up to translation) $\bepsilon-\fsl_2$-tiling of $\ZZ^3$.
    \end{theorem}
    \begin{proof}
      Assume $(a_{i_1,i_2,i_3})_{i_k\in\ZZ}$ is a $(-1,-1,-1)-\fsl_2$-tiling of $\ZZ^3$.  Pick $i_1,i_2,i_3$ with $a_{i_1,i_2,i_3}$ minimal.  Applying \eqref{eq:higher sl2 recursion} gives
      \[a_{i_1+1,i_2,i_3}a_{i_1,i_2-1,i_3}=a_{i_1,i_2,i_3}a_{i_1+1,i_2-1,i_3}+1=a_{i_1,i_2,i_3}^2+1,\]
      where we applied Lemma~\ref{le:constant slices} in the last equality.  If $a_{i_1,i_2,i_3}>1$, this implies $a_{i_1+1,i_2,i_3}<a_{i_1,i_2,i_3}$ or $a_{i_1,i_2-1,i_3}<a_{i_1,i_2,i_3}$, contradicting minimality, so we must have $a_{i_1,i_2,i_3}=1$.  This implies $\{a_{i_1+1,i_2,i_3},a_{i_1,i_2-1,i_3}\}=\{1,2\}$.  Without loss of generality we will assume $a_{i_1+1,i_2,i_3}=2$ and $i_1+i_2+i_3=1$.  Then applying \eqref{eq:higher sl2 recursion} repeatedly shows that $a_{j_1,j_2,j_3}$ with $j_1+j_2+j_3=r\ge1$ is exactly the $r^{th}$ odd Fibonacci number $F_{2r-1}$, see Example~\ref{ex:Fibonacci}.  Similarly one sees that $a_{j_1,j_2,j_3}$ with $j_1+j_2+j_3=r\le0$ is the odd Fibonacci number $F_{-2r+1}$.
    \end{proof}

    \begin{theorem}
      For $\bepsilon=(1,1,1)$, there does not exist an $\bepsilon-\fsl_2$-tiling of $\ZZ^3$.
    \end{theorem}
    \begin{proof}
      Assume $(a_{i_1,i_2,i_3})_{i_k\in\ZZ}$ is a $(1,1,1)-\fsl_2$-tiling of $\ZZ^3$.  Pick $i_1,i_2,i_3$ with $a_{i_1,i_2,i_3}$ minimal.  Applying \eqref{eq:higher sl2 recursion} gives
      \[a_{i_1+1,i_2,i_3}a_{i_1,i_2-1,i_3}=a_{i_1,i_2,i_3}a_{i_1+1,i_2-1,i_3}-1=a_{i_1,i_2,i_3}^2-1,\]
      where we applied Lemma~\ref{le:constant slices} in the last equality.  But this implies $a_{i_1+1,i_2,i_3}<a_{i_1,i_2,i_3}$ or $a_{i_1,i_2-1,i_3}<a_{i_1,i_2,i_3}$, contradicting minimality.
    \end{proof}

    \begin{corollary}
      For $n\ge4$, there is no $\bepsilon\in\{\pm1\}^n$ for which there exists an $\bepsilon-\fsl_2$-tiling of $\ZZ^n$.
    \end{corollary}
  
\end{document}